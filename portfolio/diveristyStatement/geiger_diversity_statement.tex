\documentclass{article}[12pt]

\usepackage{fullpage}
\title{Diversity Statement}
\author{Eric Geiger}
\date{}

\begin{document}
\maketitle

Mathematics is often touted as a universal language, but when the access and means to acquire fluency in that language differs wildly between schools and classrooms, teaching mathematics and fostering future mathematicians requires an understanding of the diverse backgrounds our students have.

Math is unique in that starting from grade school, courses build off of each other year to year and a student that is not successfully taught a concept in a prior class is likely to have a harder time succeeding in later classes in a snowballing effect that leaves many students, particularly those coming from disadvantaged backgrounds behind.

When public schools are funded by local property taxes it is no surprise that areas of lower wealth and historically disadvantaged communities suffer from a lack of resources.

I take special care to ensure that when I am testing students, I am testing principally on the material covered in class -- not material that student may or may not have covered years ago. For instance, a trick often used in order to have a nice integral for a computation of a function's arc-length involves quadratic factoring that usually was covered long ago.
While this provides a nice looking solution and has merit in examples, it is apt to miss the point and cause confusion particularly with students who already have a history of struggling with math.  However, this does not mean dismissing topics from previous math courses completely.  I use non-graded work, or classroom time to help identify background topics where a student may need review and extra help.

Providing clear direction to resources available on campus, like NC State's math tutorial center, is especially important in helping a student body with a diverse math background succeed in the classroom.  I always include these resources on my syllabus and emphasize their existence throughout the semester in writing and speech.

In addition to differing mathematics backgrounds, students with disabilities can suffer due to inaccessible course materials. Enhancing accessibility of my own course material is priority.  Providing audio recordings of the syllabus and lectures on a course website not only helps students with a visual disabilities, but is a boon to all students and enhances the experience for everyone.

Moreover, NC State has a Disability Resources Office (DRO) which I have pointed several students to, who were unaware that they qualified for accommodations. With all the resources available on campuses, I make sure to remind my students often of the services that they can take advantage of and recommend them to specific students as I find they are needed.

I have taught classes with as few as 40 students, and as large as 200 students.
No matter the size of the class I read NC State University's Non-Discrimination Policy, and make it clear that the classroom is a welcome space for students of all backgrounds.  I additionally include a Trans-Inclusive Statement to make clear my wish to affirm and respect the identities of transgender students, and provide a clear channel for students to let me know if they have a name or pronouns that are not reflected in the student directory.

Diversity is an important part of a college campus.  Bringing together people and ideas from all different backgrounds is beneficial to the world.
To that end it is imperative that our classrooms and research opportunities for students are a welcoming and inviting space for a diverse student body, especially when the faculty that support them are often much less diverse.
To help understand, and support the needs of students of diverse backgrounds I have participated in NC State's GLBT Advocate Program, am a member of NC State's AWM Chapter, and have completed an Accessibility in the Classroom course. 

The desire to serve a diverse campus requires continuing education and outreach. I am committed to pursue both actively and personally as an instructor, and as an advocate and participant of programs that do and do not exist yet promoting diversity.
\end{document}


% Points:
%
% Diverse backgrounds in math
% 	Helping students from backgrounds of varying levels
%	acknowledgement that often unrepresented communities suffer most 
%	How to address this:	
%		Proper testing and grading	
%		Recognizing and filling in background in way that works	
%	Disabilities	
%	
% Experience
%	Taught classrooms of varying sizes:
%	40 to > 200
%	Project SAFE
%	Lgbt advocate
%	suicide prevention
%	AWM member
%	Accessibility in the classroom
%	Diversity statement in syllabus and how I introduce it
%
% My philosophy
%	Math universal language - however not taught universally well
%	Anecdote?
%	Brother had one bad 6th grade experience that had lasting impact
%	Not Math Person
%
%
%
%
