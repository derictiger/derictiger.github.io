\documentclass{article}[12pt]

\usepackage{fullpage}
\title{Eric Geiger - Teaching Statement}
\date{}

\begin{document}
\maketitle

%-----------------------------%
The driving force behind my teaching is a desire to share the richness of mathematics.
My goal is that students, in addition to acquiring competency in a specific course, feel more confident in their future mathematical encounters whether it be in future courses, in their careers, or simply those instances where mathematics appears unexpectedly throughout our lives.
The most rewarding moments of teaching come when a student exclaims "oh, I get it." They have not only grasped a key concept, but in that moment of clarity seen something beautiful.

%-----------------------------%
The most effective way to learn mathematics is to be part of a learning community.
Facilitation of such a community is one of my goals as an instructor,
making sure students feel comfortable asking questions during and outside of lectures.
At the beginning of each semester I emphasize that our classroom is an inclusive space and point out several diversity statements placed in the course syllabus in a first step to making sure students of various backgrounds feel seen and acknowledged.
When presenting problems for students to practice during class I use this important time to traverse the room and engage directly with individuals or small groups of students.

For every student that asks a question aloud there are several more who also have the same question whether they know it or not, so it is a small tragedy when any question goes unheard.
Every class I teach I set up and moderate a forum used for course and homework discussion, often students respond to each other bringing in and sharing their own perspectives on the material.
To provide space for all questions that would go unasked, during each lecture I collect index cards that serve two purposes:
Firstly, I ask everyone to answer a lighthearted prompt or poll and report results in the next class fostering a sense of community by serving as a class-wide icebreaker.
Secondly, and most importantly, these cards are used to ask me questions a student may not feel comfortable voicing aloud during class. This provides an immediate outlet that removes barriers to important questions, and the voicing of personal concerns.

Math at its core is about problem solving and I strive to make sure that the students acquire problem solving skills from class.
When presenting problems to the class I approach the problem as if I were seeing it for the first time, making sure to be deliberate in voicing questions that the problem raises, showing them that a solution to one question is often obtained after answering smaller, unasked questions.
When presented with an overly contrived "real-life example" I do not hide the fact that the problem is not very realistic, but I ask the students to think critically about what assumptions are being made and what changes they would make.  They are often quick to point out that a differential equation that shows population growth with an initial population of one may not be realistic for a pond of fish, and allows for a natural lead in to more complicated models. 
Additionally I always provide time during lecture for students to tackle problems in groups.  If there is room I will have students work on the board in groups, in a hybrid calculus course I taught requiring board work made the students much more likely to interact and discuss.

%-----------------------------%
At NC State I have filled the role of recitation leader several times, and have been instructor of record for three classes, one of which was a flipped/hybrid style class. I have interacted with class sizes with as few as 20 students and was recognized in NC State's Thank a Teacher Program by a student in a class of 200 students.
From these positions I have been able to advise students who were interested in pursing math further, of these one is currently applying to PhD programs.

%I have served as a committee member for the NC State Math department's GIST (Graduate Instructor Support \& Tools) program which helps to organize resources and workshops for graduate instructors who may be teaching for the very first time, or might want to improve their teaching.
%I have also participated as mentor in the NC State's math department program UUG which partners undergraduate math majors with graduate mentors who provide guidance and advice while they are working towards their degree and plan for their careers after graduation.
%Additionally I have participated in teaching workshops and am working towards completing the Teaching and Communication Certificate offered by NC State.

%-----------------------------%
Non-University teaching experience has included teaching a 3-week course with over 100 hours of class time called "Paradoxes and Infinities" through the Johns Hopkins Center for Talented Youth program.
While there were suggested topics there were no pre-designed lesson plans and I had complete freedom in developing the curriculum for the class. I taught foundations of set theory through the physical explorations of shapes made of construction paper and used readings ranging from Lewis Carroll to Jorge Borges to illustrate and introduce topics before approaching them mathematically.

%-----------------------------%
Through my diverse teaching history I have had the opportunity to explore a varied range of teaching and am interested in using my experience to design interesting courses that could serve as a gateway for undergraduate research, and participate in the maintenance and restructuring of current courses offered by the University of Minnesota.

\end{document}
