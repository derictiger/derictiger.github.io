\documentclass[]{beamer}
\usetheme{Madrid}
\usecolortheme{beaver}
\usepackage[utf8]{inputenc}
\usepackage{amsmath}
\newcommand{\R}{\mathbb{R}}
\usepackage[absolute,overlay]{textpos}
\usepackage{pdfpages}
\usepackage{animate}
\usepackage{xcolor}

\title{Geometric Methods in Computer Vision}
\author{Jose Agudelo, Brooke Dippold, Ian Klein, Alex Kokot}
\institute{
Mentors: Eric Geiger and Irina Kogan
%This REU project was supported by the NCSU Department of Mathematics and the NSA REU grant
%(PIs: Mette Olufsen and Seth Sullivant).
}
\date{July 30, 2020}

\begin{document}

\frame{\titlepage}

\begin{frame}{Congruence and Curvature}
\begin{itemize}
    \item What makes objects congruent?
    \item Transformation groups
    \begin{itemize}
        \item Special Euclidean group
        \item Equi-affine group
    \end{itemize}
    \item Invariants - properties of curves which do not change under transformation
    \begin{itemize}
        \item Euclidean curvature: $\kappa$
        \item Equi-affine curvature: $\mu$
    \end{itemize}
\begin{center}
\includegraphics[width = 8cm]{tri.png}
\end{center}
\end{itemize}
    
\end{frame}

\begin{frame}{Euclidean versus Equiaffine Invariants}

$\gamma = x(t), y(t)$ is a parametric curve.  
\vspace{0.5cm}
\begin{center}
\begin{tabular}[h]{|c|c|}
\hline 
    Euclidean & Equiaffine \\
    \hline 
       & \\
    $s(t) = \int_{0}^{t} |\gamma_{\tau} | d\tau $ & $\alpha(t) = \int_{0}^{t} (\gamma_\tau (\tau) \times \gamma_{\tau\tau}(\tau))^{\frac{1}{3}} d\tau$ \\
       & \\
   curvature:  $\kappa(s) = \gamma{_s_s}$ &curvature: $\mu(\alpha) = \gamma {_\alpha _\alpha _ \alpha} \times \gamma {_\alpha_\alpha} $\\
       & \\
    $T' = \kappa(s) N$ & $T'' = \mu(\alpha) T$\\ 
       & \\
    $\gamma'(s) = T(s)$ & $\gamma'(\alpha) = T(\alpha)$\\
\hline 
\end{tabular}
\end{center}
The last two equation in each case allow reconstruction of $\gamma$ from curvatures. 
\end{frame}

\begin{frame}{Power Series Method for Solving $T'' = \mu(\alpha) T$}
\begin{itemize}
    \item $\mu(\alpha)=$ $\sum\limits_{n=0}^{\infty}  g_n\alpha^n$  
    \begin{itemize}
        \item $\mu(\alpha)$ must be analytic
        \item all  $g_n$'s are known
    \end{itemize}
    \item $T= \sum\limits_{n=0}^{\infty}  b_n\alpha^n$ 
    \item $T''= \sum\limits_{n=2}^{\infty} n (n-1) b_n \alpha^{n-2}$
    \item By substituting the above series into $T'' = \mu(\alpha)T$, you can solve for the $b_n$ coefficients that make up the series solution to $T$  
    \item Integrating $T$ results in $\gamma$
\end{itemize}
\end{frame}


\begin{frame}{Curves $\gamma$ Reconstructed from Different $\mu(\alpha)$ }
\begin{columns}

\begin{column}{0.5\textwidth}
\begin{center}
\includegraphics[width=3.95cm]{alpha.png}\\{$\mu(\alpha) = \alpha$}\\
\includegraphics[width=3.95cm]{newalpha2.png}\\{$\mu(\alpha) = \alpha ^2$}\\
\end{center}
\end{column}

\begin{column}{0.5\textwidth}
\begin{center}
\includegraphics[width=3.95cm]{-alpha.png}\\{$\mu(\alpha) = - \alpha$   }\\
\includegraphics[width=3.95cm]{-alphasquared.png}\\{$\mu(\alpha) = - \alpha ^2$}\\
\end{center}
\end{column}
\end{columns}
\end{frame}

\begin{frame}{Picard Iteration for $\mu(\alpha)=-1$}
Recall that for constant $\mu(\alpha)<0$, the reconstructed curve becomes an ellipse. For $\mu(\alpha)=-1$, and $\begin{bmatrix}
T\\
N
\end{bmatrix}_0=\begin{bmatrix}
1&0\\
0&1
\end{bmatrix}$ the Picard iterations approximate to a circle in the following way:
\begin{center}
\animategraphics[loop,autoplay,width=5cm]{3}{PIcircle}{1}{20}
\end{center}
\end{frame}
\begin{frame}{What Picard Iterations Tell Us}
    As we continue to iterate our system, the iterated curves converge to our solution reconstruction. Similarly, our parameterization converges to the solution parameterization. 
\begin{columns}
\column{8cm}
\animategraphics[loop,autoplay,width=5cm]{1}{PIexample}{1}{4}
\column{3cm}
\begin{itemize}
\color{blue}
\item $\gamma_1(t)$
\color{red}
\item $\gamma_6(t)$
\color{green}
\item $\gamma_{12}(t)$
\color{purple}
\item $\gamma_{18}(t)$
\end{itemize}


\end{columns}    

\end{frame}



\begin{frame}{Euclidean Signature}

\begin{textblock*}{5cm}(4.5cm,3cm) % {block width} (coords)
\scalebox{1.5}{
$\{(\kappa(s), \kappa'(s))\}$}
\end{textblock*}

\begin{textblock*}{7cm}(1.25cm,4.75cm) % {block width} (coords)
\includegraphics[width=5cm]{SignatureExample.png}
\end{textblock*}

\begin{textblock*}{4cm}(8cm,4.5cm) % {block width} (coords)
\scalebox{.75}{
\includegraphics[width=4cm]{ThumbPrint.jpg}}
\end{textblock*}

\end{frame}

\begin{frame}{Data is Noisy}

\begin{textblock*}{4cm}(3.5cm,3.5cm) % {block width} (coords)

\includegraphics[width=6cm]{DataIsNoisy.png}
\end{textblock*}
    
\end{frame}

\section{A Metric On Signatures}

\begin{frame}{The Hausdorff Metric}

\begin{textblock*}{4cm}(1.75cm,3.6cm) % {block width} (coords)

\includegraphics[width=9cm]{HausdorffPicture.png}
\end{textblock*}
    
\end{frame}



\begin{frame}{Signature as a Phase Portrait}
    
    \begin{textblock*}{4cm}(3.8cm,2.6cm) % {block width} (coords)

\includegraphics[width=5.5cm]{UVectorField.png}
\end{textblock*}
    
\end{frame}
\section{From Signatures to Curves}

\begin{frame}{The Tube Neighborhood}

\begin{textblock*}{4cm}(1.75cm,2.6cm) % {block width} (coords)

\includegraphics[width=9cm]{TubeDetailed.png}
\end{textblock*}
    
\end{frame}


\begin{frame}{In Practice}
    
\begin{textblock*}{4cm}(2.4cm,1cm) % {block width} (coords)

\includegraphics[width=8cm]{SignatureSqueeze.png}

\end{textblock*}
    
\end{frame}

\begin{frame}{Smooth Deformations of Curvature}
    \begin{itemize}
        \item We can smoothly deform curvatures by adding a "bump" function
        \begin{itemize}
         
        \end{itemize}
        
        \item Use bump functions to construct $\kappa_n$ such that $|\sin(s) - \kappa_n(s)| \leq \frac{\pi}{n}$ for all s and $\kappa_n$ is smooth and periodic
        \item $\kappa_n(s)$ converges uniformly to $\sin(s)$
        
         \centering{\includegraphics[width=1.4in]{bumpedSin,3,3.pdf}
         \caption{\kappa_3}
          \hspace{.5in}
          \includegraphics[width=1.4in]{bumpedSin,3,10.pdf}
          \caption{\kappa_{10}}}
    
         \item Call $\Gamma_\kappa$ the curve reconstructed from curvature $\kappa(s)$ with initial conditions $\alpha_0 = x_0 = y_0 = 0$. 
        \end{itemize}
   
\end{frame}
\begin{frame}{Reconstructions of $\sin(s)$ and $\kappa_3(s)$ on $[0,12\pi]$}
    \centering
    \includegraphics[width = 1.8in]{Reconstructed From Sine Curvature.pdf}
    \caption{\Gamma_{\sin}}
    \includegraphics[width=2in]{n = 3.pdf}
    \caption{\Gamma_{\kappa_3}}
    
\end{frame}
\begin{frame}{Reconstructions of $\kappa_5(s)$ and $\kappa_{10}(s)$}
     \centering
    \includegraphics[width = 1.8in]{n = 5.pdf}
    \caption{\Gamma_{\kappa_5}}
    \includegraphics[width=2in]{n = 10.pdf}
    \caption{\Gamma_{\kappa_{10}}}
\end{frame}
\begin{frame}{Noninteger n}
\centering
    \includegraphics[width = 1.8in]{n = 3,7.pdf}
    \caption{\Gamma_{\kappa_{3/7}}}
    \includegraphics[width=2in]{n = 4,5.pdf}
    \caption{\Gamma_{\kappa_{4/5}}}
    
    
\end{frame}



\end{document}
